
\documentclass[numerado]{plantillasEPS} % Las opciones son: numerado y serif

\title{Acta de la Junta de Centro de la Escuela Politécnica Superior} % Título del acta

\begin{document}

El 2 de julio de 2019 a las 11:00 tuvo lugar una Junta de Centro Ordinaria de la Escuela Politécnica Superior, con la asistencia de: 

% El entorno "asisten" es un entorno especial para indicar la asistencia a un acto. Entre corchetes la descripción y después los nombres.

\begin{asisten}
    \item[Director:] Martínez Sánchez, Jose María
    \item[Secretaria Académica:] Rodríguez Marín, Pilar
    \item[Administradora Gerente:] Checa Martínez, Victoria
    
    \item[Subdirecciones:] 
    \begin{asisten}
        \item[Profesorado:] Alamán Roldán, Xavier
        \item[Estudiantes:] Alarcón Rodríguez, Miren Idoia
        \item[Calidad de las Enseñanzas:] Anguiano Rey, Eloy
        \item[Asuntos Económicos e Infraestructura:] González de Rivera Peces, Guillermo
        \item[Investigación e Innovación:] González Rodríguez, Joaquín
        \item[Estudios de Posgrado y Formación Continua:] López de Vergara Méndez, Jorge E.
        \item[Estudios de Grado:] Moreno Llorena, Jaime
    \end{asisten} 
    \item[Directores de Dptos. Integrados en la EPS:] 
    \begin{asisten}
        \item[Dpto. Ingeniería Informática:] Díez Rubio, Fernando
        \item[Dpto. Tecnología Electrónica y de las Comunicaciones:] Bescós Cano, Jesús
    \end{asisten} 
    \item[Directores de Dptos. no Integrados en la EPS con 12 o más ECTS impartidos:] 
    \begin{asisten}
        \item[Dpto. Matemáticas:] Barceló Taberner, Bartolomé
        \item[Dpto. Física Aplicada:] Castaño Palazón, José Luis (en nombre del/de la Director/a del mismo)
    \end{asisten} 
    \item[Representantes de Profesores/as Permanentes:] 
    Martínez Muñoz, Gonzalo; Rodríguez Ortíz, Francisco de Borja; Ruíz Cruz, Jorge Alfonso; Suárez González, Alberto
    \item[Representantes de Profesores/as e Investigadores/as con Vinculación no  Permanente:] Jurado Monroy, Francisco
    \item[Representantes de Personal Docente e Investigador en Formación:] Polo López, Lucas; Reyes Sánchez, Manuel
    \item[Representantes de Estudiantes:] Muñoz Murillo, Álvaro
    \item[Asiste como invitada:] Asperilla Ruiz, Rocío
\end{asisten}

\newpage

\chapter{Con el siguiente Orden del Día:}

\section{Se aprueba por asentimiento el acta correspondiente a la Junta Ordinaria de 31 de mayo de 2019.} 
\section{Informe del Director}
El Director informa sobre varios asuntos de distinta índole:

\begin{itemize}
    \item Concesión del Proyecto de Red de Universidades Europeas CIVIS en el que participa la UAM por parte de la UE: se han concedido 18 proyectos de 43 solicitados. 
    \item Conclusión  de una  auditoría  de  los espacios  de  la  EPS  sobre  los  que  el  equipo  de dirección    tiene    delegada    la    gestión.    En    este    sentido    se    ha    realizado la correspondiente   delegación   a   los departamentos   para su   gestión;   los   cuales disponen  de  toda la  información  sobre  la  ocupación de  los  mismo.  Se  prevé  la revisión de la situación en torno a un año. 
    \item Aprobación  en  CdG de  Solicitud  de  Título  de  Grado  en  Ingeniería  Biomédica,  así como de la Memoria de Verificación del Máster en Ciencia de Datos. 
    \item Conclusión  de  un  primer  borrador  sobre  el  grado  en Ingeniería  de  Datos  por  parte del  grupo  de  trabajo  al  que  se  le  asignó  dicha  tarea.  Dicho  borrador se  envía  a  los directores de los departamentos implicados en dicho título para recabar su opinión antes de continuar con su desarrollo.
    \item Instauración  de  un  nuevo  protocolo  para la contratación de  personal con  cargo  a proyectos gestionados por el Servicio de Investigación.
    \item Establecimiento  de  dos convenios  EPS:  AULA  UAM-IIC en “Programación software para aplicaciones críticas” y Convenio Retos UAM-Capgemini para TIC en la EPS.
    \item En lo que se refiere a actos pasados, el Director informa acerca de los siguientes:
    
    \begin{itemize}
        \item Entrega  de  los  Premios  Alumni  (7  junio).  En  este  sentido,  agradece  a  Pilar Rodríguez la presentación de los premiados.
        \item Jornada  STEM-Oracle  (8  de  junio).  En  este  sentido,  agradece  a  Guillermo González de Rivera Peces la organización de dichas jornadas.
        \item Campus Quiero Ser Ingeniera, QSI (10-14 junio). Ángel de Castro la dirección del   mismo,   así   como a   todos   los   participantes   en   las   diversas   mesas redondas y en los tres talleres que se realizaron.
        \item VII  Jornadas  de  Iniciación  a  la  Universidad  (24-28  junio).  En  este  sentido, agradece a  Guillermo  González  de  Rivera  Peces  su  dirección,  así  como  a todos  los  participantes  en  los  dos  talleres  y  en  las  diversas  actividades realizadas (visitas al campus y a la EPS). Además, agradece especialmente su apoyo  y  colaboración  a Álvaro  Muñoz,  representante  del  estamento  de estudiantes  en  la  JdC  de  la  EPS. Por  último,  agradece  a  la  Vicerrectora  de Estudiantes y Empleabilidad por su apoyo y asistencia al acto de clausura de estas jornadas.
    \end{itemize}
    
    \item En lo relativo a actos futuros, el Director informa acerca de los siguientes:
    
    \begin{itemize}
        \item Congreso  de  Antropología  (semana  8-12  julio),  que  reunirá  en  torno  a  800 personas  en  el  Edificio  A  de  la  EPS.  En  este  sentido,  desea  expresar  su agradecimiento  expreso  a  Victoria  Checa  por  el  esfuerzo  de  organización que ha requerido albergar dicho congreso.
        \item Acto de bienvenida para estudiantes de nuevo ingreso en la EPS en el curso 2019/20 (6 septiembre, a las11:00)
        \item Acto/jornada  de  bienvenida  para estudiantes  de  nuevo  ingreso en  la UAM en  el curso 2019/20  (12  septiembre,  de12:00a  17:00).  Dicho  día  no  habrá clases  para  los  estudiantes  de  primer  curso en  el  periodo  dedicado  a  la jornada
        \item Diálogos  de  la  Cultura,  en  los que  José  Luis  Crespo  de  QuantumFracture dialogará con Xavier Almán sobre Tecnologías Cuánticas (27 de septiembre, 12:00)
    \end{itemize}    
    
    \item El Director también informa sobre la realización, en el momento en que tiene lugar esta JdC, del Congreso IbPRIA2019 (1-4 Julio). En este sentido, agradece a Julián Fiérrez, así como a toda lo organización del mismo, la invitación al PDI y PDIF de la EPS para poder asistir a las presentaciones.
\end{itemize}

\section{Asuntos de Estudios de Posgrado y Formación Continua} 
\subsection{Se  aprueba  por  asentimiento:}  la  Memoria  de  verificación  del  Máster  Universitario  en \textit{"Deep Learning for Audio and Video Signal Processing"}

Se  informa  sobre  un  título  presentado  recientemente  por  la  F.  de  CC.  EE.  Y  EE., denominado Máster en Economía Cuantitativa, y para el que solicitan colaboración de la EPS  en  BigData  (4  ECTS).  La  información  correspondiente  se  ha  trasladado  a  los directores de los departamentos de la EPS.

\section{Asuntos de Profesorado}
\subsection{Se aprueban por asentimiento:} los asuntos de profesorado presentados a propuesta de los Departamentos.

\section{Asuntos de Internacionalización}
El Subdirector ha excusado su asistencia a esta JdC. 

El  Director  informa  sobre  la  apertura  del  Plan  ADid,  similar  al  Plan  Doing  enfocado  al PAS. Informa también de la reciente visita, el pasado 25 de junio, del Polytech Marseille (Aix-Marseiile),   miembro   de   la   red   CIVIS,   para   estudiar   posibles   colaboraciones (Erasmus, p.e.)
\section{Asuntos de Investigación e Innovación}
No hay asuntos que tratar. 
\section{Asuntos Económicos e Infraestructura}
El  Subdirector  informa  sobre  el  próximo  corte  de  puertos  por  parte  de  Tecnologías  de Información por problemas de ataques informáticos a la universidad.

Ante las preguntas de miembros de la JdC, el Subdirector confirma que se podrá seguir utilizando el escritorio remoto a través de VPN y que se espera que se realice el corte en torno  al  verano.  Otras  preguntas,  sobre  cuyas respuestas  el  subdirector  no  tiene información, se trasladan al punto de la JdC de Ruegos y Preguntas. 

\section{Asuntos de Calidad de las Enseñanzas}

No hay asuntos que tratar. 

\section{Asuntos de Estudiantes y Empleabilidad}
La  Subdirectora  informa  que  se  han  resuelto favorablemente  todas  las  solicitudes  de compensación   curricular   presentadas   por   estudiantes   de   la   EPS   en   la   convocatoria ordinaria.  También  informa  de  que  se  está  trabajando  en  una  nueva  normativa  de permanencia en la correspondiente comisión de la UAM.

\section{Asuntos de Estudios de Grado}
\subsection{Se aprueba por asentimiento:} el Calendario de Horarios Reservados para los Grados de la EPS en el primer cuatrimestre del curso 2019/20
\subsection{Se   aprueba   por   mayoría:} la  consideración  de  los  “Periodos  sin  actividad docente” establecidos entre el primer y el segundo cuatrimestre en los Calendarios de Clases y Evaluación de la UAM de cada cursocomoválidos para realizar revisiones de exámenes. 

Los resultados de la votación fueron los siguientes: 11 votos a favor, 5 votos en contra y 4 abstenciones. 

Antes  de  la  votación, Álvaro  Muñoz,  representante  del  estamento  de  estudiantes, indica su voluntad de votar en contra de la propuesta, aun entendiendo que quizá sea un  mal menor  dados  los  problemas  ocasionados  en  el  caso  de  algunas  asignaturas. Indica  que,  dado  que  las vacaciones  de Navidad  no  son tales  debido  a  la  inminencia del  periodo  de  exámenes  tras  las  mismas,  este  periodo  sin  actividad  docente  se utilizaba como periodo de descanso tras la realización de los exámenes. 

Tras algunas intervenciones, la discusión derivada se traslada al apartado de Ruegos y Preguntas de la presente acta. 

\subsection{Se   aprueba   por   asentimiento:}   la Memoria   de   Verificación   del   Grado   en Ingeniería  Biomédica tal  como  está  disponible  en  la  documentación  presentada  a  la JdC pero con la siguiente consideración:dado que se trata de un documento en el que aún  están  trabajando  los  centros  involucrados,  la  JdC  otorga  la  confianza  a  las comisiones   que   trabajan en   la   elaboración   de   la   memoria   para   realizar   las modificaciones  que  se  consideren  pertinentes  con  el  objetivo  de  concluir  la  misma con éxito antes de finales de septiembre. 

\section{Ruegos y Preguntas}

Se  traslada  a  este  punto  una  pregunta  realizada  por  Eloy  Anguiano  al  Subdirector  de Económicos  e  Infraestructura respecto  a  la  permanencia  de  los  servicios  de  http  y  sftp tras el cierre de los puertos establecido por TI.Igualmente,  se  traslada  a  este  punto  la  inquietud  derivada  de  la discusión  mantenida durante  el  debate  sucedido  en  torno  al  punto  10.2.;  esto  es,  la  consideración  de  los “Periodos sin actividad docente” establecidos entre el primer y el segundo cuatrimestre en  los  Calendarios  de  Clases  y  Evaluación  de  la  UAM  de  cada  curso  como  válidos  para realizar  revisiones  de  exámenes. En  este  sentido,  el Director  propone que  este  asunto se vuelva a tratar en la comisión pertinente a la vuelta del verano. 

Sin más asuntos que tratar, se levanta la sesión a las 12:05


% El primer parámetro es optativo es el ancho de la firma. Si no se pone o se deja en blanco: 5cm
% El segundo es parámetro optativo es la distancia del texto a la firma. Si no se pone: en la parte baja de la última página. Se puede no poner pero no se puede dejar vacío.
% El tercer parámetro es obligatorio y es el nombre del firmante.
% El cuarto parámetro es optativo y es la altura de la imagen o si no se especifica imagen el espacio dejado para la firma. Default: 2cm.
% El quinto parámetro es optativo y es el nombre de la imagen con la firma sin extensión. Si no se pone o se deja vacío es sin imagen de firma.
% El primer parámetro optativo es obligatorio ponerlo aunque sea vacío si se quiere indicar el segundo.
% El cuarto parámetro optativo es obligatorio ponerlo aunque sea vacío si se quiere indicar el quinto.
\firmado[6cm][2cm]{Pablo Ruiz Picasso\\Pintor a tiempo completo}[3cm][firma-picasso]

\end{document} 